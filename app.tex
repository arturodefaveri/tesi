\chapter[Applications]{Applications}

\section{The Spectrum Problem}
\begin{defn} A subset $S$ of the natural numbers is called a \textbf{spectrum} if there is a vocabulary $\tau$ and a $\tau$-sentence $\phi$ such that 
$S=\{ \norm{M} : M \models \phi\} $.
Given a $\tau$-sentence $\phi$ we shall write $\Spec(\phi)$ for the spectrum of $\phi$.
\end{defn}

For example, let $\tau=\emptyset$. Taking disjunctions of the sentences `there are exactly $n$ elements' every nonempty finite subset is a spectrum. 
Let $\tau=\{R\}$, $R$ binary, and $\psi$ be the conjunction of the axioms for the theory of equivalence relations with the sentence 

$$\forall x_1 \exists x_2 \ldots \exists x_m \bigwedge_{i < j } (x_i \neq x_j \land Rx_ix_j) \land \lnot \exists x_1 \ldots \exists x_{m+1} \bigwedge_{i < j } (x_i \neq x_j \land Rx_ix_j) \text{,}$$

stating that each equivalence class contains exactly $m$ elements. 
For each $m \ge 1$ let $S$ be the set of all the positive numbers which are divisible by $m$. 
Then $S= \Spec(\psi)$.  

Using the results proved in the first chapter, we obtain the following. 

\begin{thm} 
For any FO $\tau$-sentence $\phi$, $\Spec(\phi)$ or $\Spec(\lnot \phi)$ is cofinite. 
\begin{proof} 
Let $\Sigma$ be the set of sentences with (i) `there are at least $n$ elements' ($n \ge 1$), (ii) $\forall x_1 \ldots \forall x_k R x_1 \ldots x_k $ for every $k$-ary $R$ in the vocabulary $ \tau$. 
$\Sigma$ is satisfiable. 
Any two models of $\Sigma$ $\mathcal{A}$ and $\mathcal{B}$ are $\omega$-isomorphic via the set $I:=\{ p \in \Part(\mathcal{A}, \mathcal{B}): \norm{\dom(p)} < \infty  \}$, hence elementarily equivalent. 
Therefore the theory $\Sigma$ is complete.
Let us say $\Sigma \models \phi$. 
By compactness there exists a finite $\Sigma_0 \subset \Sigma$ such that $\Sigma_0 \models \phi$. 
Let $n_0$ be larger than than the $n$ such that the sentences (i) are in $\Sigma_0$.
Then $\phi$ has a model of cardinality $n$ for every $n \ge n_0$. 
\end{proof}
\end{thm}

We could have proved the above theorem by means of the 0-1 law: if $\lambda(\phi)=1$ then there exists $k$ such that for every $n$ greater than $k$ $\lambda_n(\phi)>0$.
Hence $\phi$ has a model of cardinality $n$, for each $n > k$. 
Similarly if $\lambda(\lnot \phi)=1$. 

In what follows we restrict our investigation to certain interesting vocabularies. 

\subsubsection{Finite Number of Unary Relation Symbols} 

We start with a rather simple example. 
Let $\tau=\{P_1, \ldots , P_r\}$ with $P_i$ unary. 
Recall, by Example \ref{unary}, that every $\tau$-sentence is equivalent to the negation or boolean combination of sentences of the form $\exists ^{=l} x P^{\alpha} x$. 
Therefore 

\begin{thm} 
The spectra involving the vocabulary $\{P_1, \ldots , P_r\}$ are precisely the finite and cofinite sets. 
\end{thm} 

\begin{rem} Given a $\sigma$-structure $\mathcal{A}$, a subset $S \subseteq A$ is elementarily \textbf{definable} in $\mathcal{A}$ if there is a formula $\zeta \in L_1[\sigma]$ such that for all $a \in A$, $\mathcal{A} \models \zeta[a] $ iff $a \in S$. 
It's well known that the finite and cofinite sets are exactly those definable in $(\mathbb{N}, <)$. 
Therefore: $S$ is the spectrum of a $\sigma$-sentence iff $S$ is definable in $(\mathbb{N}, <)$.
\end{rem}

\subsubsection{A Single Unary Function Symbol}
Spectra involving the vocabulary $\{R\}$, where $R$ is a binary relation symbol, are fairly complex. 
An intermediate case is given by $\{f\}$, where $f$ is a unary function symbol. 
We'll show that spectra involving this vocabulary are exactly the ultimately periodic sets. 

Remember that a set $S$  is \textbf{ultimately periodic} if there exist $k$ and $p >0$ such that $n \in S$ iff $n+p\in S$, when $n >k$; remember further that an \textbf{arithmetic sequence} is a sequence $(a_n)_n \in \omega^\omega$ of the form $(m+dn)_n$, where $d$ and $m$ are fixed natural numbers. 

\begin{lem} 
\label{lemmaf1} 
$S$ is ultimately periodic iff it is the finite union of arithmetic sequences.
\begin{proof} 
If $S$ is the union of arithmetic sequences $(m_i +d_i n)_n$, $i \in I$, with $I$ finite, then define $k$ to be $\max\{m_i : i \in I\}$ and $p$ to be the least common multiple of $d_i$. 

Conversely, if $S$ is an ultimately periodic set, then $S$ is made up by the set $\{n \le k : n \in S\}$ (which is the union of trivial arithmetic sequences with $d=0$) and the union of the arithmetic sequences $(m_i + d_i n)_n$ with $k < m_i \le k+p$ and $m_i \in S$. 
\end{proof}
\end{lem}

\begin{lem} \label{lemmaf2} 
If $S$ is a spectrum involving the vocabulary $\{f\}$ then there are numbers $D$, $k$ such that for each $m \in S$, $m>k$, there is $d$ with $0<d\le D$ such that the sequence $(m+dn)_n \subseteq S$. 

\begin{proof}
Assume that $S=\Spec(\phi)$ and that $\qr(\phi)=r$. 
Let $\mathcal{A}$ be a $\{f\}$-structure such that $\mathcal{A} \models \phi$ (i.e. $\norm{A} \in S$) and $\norm{A} \gg 2^{r}$\footnote{This number has to depend only on $r$.}. 
Put an arrow from $a_i$ to $a_j$ just in case $a_j=f(a_i)$. Then $\mathcal{A}$ is a digraph. 
Each connected component consists of disjoint trees whose roots\footnote{The \textbf{root} is the vertex of the tree than can be reached by all the other vertices. The natural orientation is thus \emph{towards} the root.}
form a cycle. 
For clarity's sake, an example of such a connected component is depicted in Figure \ref{graph}.

\begin{figure}
\centering
\usetikzlibrary{arrows.meta}
\begin{tikzpicture}
\label{graph}
\draw[{Circle}-Latex] (0.866,-0.5) -- (0,0); 
\draw[{Circle}-Latex] (1.831,-0.241) -- (0.866,-0.5);
\draw[{Circle}-Latex] (1.731,-1) -- (0.866,-0.5);
\draw[{Circle}-Latex] (1.124,-1.465) -- (0.866,-0.5);
\draw[{Circle}-Latex] (0,1) -- (0,0);
\draw[{Circle}-Latex] (-0.866,1.5) -- (0,1);
\draw[{Circle}-Latex] (-1.732,1) -- (-0.866,1.5);
\draw[{Circle}-Latex] (-1.732,0) -- (-1.732,1);
\draw[{Circle}-Latex] (-0.866,-0.5) -- (-1.732,0);
\draw[{Circle}-Latex] (0,0) -- (-0.866,-0.5);
\draw[{Circle}-Latex] (-2.732,1) -- (-1.732,1);
\draw[{Circle}-Latex] (-2.232,1.866) -- (-1.732,1);
\end{tikzpicture}

\caption{An example of $\{f\}$-structure.}
\end{figure}

We show that there is a $\{f\}$-structure $\mathcal{B}$ such that $\norm{A}=\norm{B}$, $\mathcal{A} \equiv_r \mathcal{B}$ and: (i) each tree in $\mathcal{B}$ has depth smaller than a constant; (ii) each cycle has length smaller than a constant. 

\begin{claim} There is a constant $c_1$ such that for every tree $\mathcal{T}$ there is a $\{f\}$-structure $\mathcal{B}_1$ such that $\norm{B_1}=\norm{T}$ and: (a) $\mathcal{B}_1 \equiv_r \mathcal{T}$, (b) the trees of $\mathcal{B}_1$ have depth\footnote{We shall call \textbf{depth} the length of the longest path starting from the root and arriving at a terminal vertex.} at most $c_1$.
\begin{proof} 
For every $s$ the set of $r$-types $\{\phi^{r} _{\mathbf{t}_s } : \mathbf{t}_s \in T^s\}$ is finite. Let $$v:=\norm{\{\phi^{r-1} _{t} : t \in T\}}\text{.}$$ 
Let $c_1:=(2^r +1)^v +1$. 
If the tree $\mathcal{T}$ has depth at most $c_1$ we are done. 
Otherwise there is a branch of $\mathcal{T}$ on which $t_2, t_4$ lie and such that $\phi^{r-1} _{t_2} = \phi^{r-1} _{t_4}$ and $d(t_2, t_4) > 2^r$. 
Suppose without loss of generality that there are $t_1$, $t_3$ such that there is an arrow from $t_1$ to $t_2$ and from $t_3$ to $t_4$. 
(If not we can simply interchange the role of the $t_j$). 
By construction $\phi^{r-2} _{t_1 t_2} = \phi^{r-2} _{t_3 t_4}$. 
We cut the arrows from $t_1$ to $t_2$ and from $t_3$ to $t_4$; we add arrows from $t_1$ to $t_4$ and from $t_3$ to $t_2$. 
This procedure creates a cycle of length at least $2^r$. 
Let $\mathcal{B}_1$ be the structure obtained. 
Clearly $\norm{B_1}=\norm{T}$ and the trees of $\mathcal{B}_1$ have depth at most $c_1$. 
By Examples \ref{acyclic} and \ref{even_order} $\mathcal{B}_1 \equiv_r \mathcal{T}$. 
\end{proof}
\end{claim}

\begin{claim}
There is a constant $c_2$ such that for every cycle $\mathcal{D}$ there is a $\{f\}$-structure $\mathcal{B}_2$ such that $\norm{B_2}=\norm{D}$ and: (a) $\mathcal{B}_2 \equiv_r \mathcal{D}$, (b) the cycles of $\mathcal{B}_2$ have length at most $c_2$. 
\begin{proof} 
For every $s$ the set of $r$-types $\{\phi^{r} _{\mathbf{d}_s } : \mathbf{d}_s \in D^s\}$ is finite. 
Let $$u:=\norm{\{\phi^{r-1} _{d} : d \in D\}}\text{.}$$ 
Let $c_2:=(2^r +1)^u +2^r +1$. 
If the cycle $\mathcal{D}$ has length at most $c_2$ we are done. 
Otherwise there are $d_1, d_3 \in D$ such that $\phi^{r-1} _{d_1} = \phi^{r-1} _{d_3}$ and $d(d_1, d_3) > 2^r +1$. 
Let $d_2, d_4$ the elements such that there is an arrow from $d_1 $ to $d_2 $ and from $d_3 $ to $d_4$. 
We cut such arrows and we add arrows from $d_1$ to $d_4$ and from $d_3$ to $d_2$. This procedure creates two shorter cycles.
Let $\mathcal{B}_2$ be the structure obtained. 
Clearly $\norm{B_2}=\norm{D}$ and the cycles of $\mathcal{B}_2$ have length at most $c_2$. 
By Example \ref{connected} $\mathcal{B}_2 \equiv_r \mathcal{D}$. 
\end{proof}
\end{claim}

Since the trees of $\mathcal{B}$ are no deeper than $c_1$ and the cycles of $\mathcal{B}$ are no longer than $c_2$, then there is a constant $c$ such that either
\begin{itemize}
\item Every connected component is smaller than $c$. In this case there is $c_3$ such that if $\norm{B} >c_3$, then $\mathcal{B}$ contains $r$ isomorphic connected components. 
Let $D:=c$ and $d$ be the size of one of these connected components. For every $m$ sufficiently large, if $m \in S$, then, for any $n$, $(m+dn) \in S$, since adding copies of one of these connected components does not change the $r$-equivalence class of $\mathcal{B}$. Or
\item A connected component is bigger than $c$. 
In this case there is $c_4$ such that if $\norm{B} >c_4$, then $\mathcal{B}$ contains $r$ subtrees that are $r$-isomorphic and whose size is bounded by a constant. 
Let $D$ be this constant and $d$ be the size of one of these subtrees. For every $m$ sufficiently large, if $m \in S$, then, for any $n$, $(m+dn) \in S$, since adding copies of one of these subtrees does not change the $r$-equivalence class of $\mathcal{B}$. \qedhere
\end{itemize}
\end{proof}

\end{lem}

\begin{thm} 
\label{spectrum}
$S$ is a spectrum involving the vocabulary $\{f\}$ iff it is ultimately periodic. 
\begin{proof} 
Let $S=\Spec(\psi)$ and $r=\qr(\psi)$. Let $k, D$ as in the statement of Lemma \ref{lemmaf2}
and let $0 < d \le D$. Let $j$ be such that 
\begin{equation} \label{good}
\text{$0\le j<d$ and there is $m>k$ with $m \equiv j$ mod $d$ and $(m+dn)_n \subseteq S$.}
\end{equation}
If $j$ satisfies \ref{good} we define $m(d,j):=\min\{m : m \equiv j \text{ mod } d \text{ and } (m+dn)_n \subseteq S\}$. 
We prove that the elements of $S$ greater than $k$ are union of sets of the form $(m(d,j)+dn)_n$ such that $0 < d \le D$ and $j$ satisfies \ref{good}. 
This implies that $S$ is the finite union of arithmetic sequences. 
By definition $(m(d,j)+dn)_n \subseteq S$. If $q > k$ and $q \in S$ we have to find $d, j$ such that $0<d\le D$; $j$ satisfies \ref{good}; $q \in (m(d,j)+dn)_n$. 
Let $d$ as in the statement of the Lemma \ref{lemmaf2}, so that $(m+dn)_n \subseteq S$. 
Let $j$ satisfy \ref{good}. $(m(d,j)+dn)_n  \equiv j$ mod $d$ and thus $m(d,j) \le q$, being the least such element. 
But $m(d,j) \equiv q$ mod $d$, and therefore $(m+dn)_n \subseteq (m(d,j)+dn)_n$. Since $q \in (m+dn)_n$, $q \in (m(d,j)+dn)_n$. 

Conversely, let $m$, $d \in \mathbb{N}$. Let $S$ be the image of the sequence $(m+dn)_n$. 
Then $S=\Spec(\psi)$ where $\psi$ is 
\begin{equation} \label{psi} \exists x_1 \ldots \exists x_m \forall x \left ( \bigwedge_{i<j} x_i \neq x_j \land  \bigvee_i x=x_i \lor \bigwedge_{i<d}  f^i x \neq x \land f^d x=x  \right )
\end{equation}
which says that except for at most $m$ points, the remaining lie on cycles of length $d$. 
This shows that an arithmetic sequence is the spectrum of a FO $\{f\}$-sentence. Then apply Lemma \ref{lemmaf1}
using disjunctions of sentences of the form \ref{psi}.
\end{proof}
\end{thm}

The proof of Theorem \ref{spectrum} provides an alternative sentence witnessing that the set of positive integers divisible by $m$ is a spectrum.
This set is clearly periodic. 

\begin{rem} It is known that the sets definable in $(\mathbb{N}, +)$, Presburger arithmetic, are exactly the ultimately periodic sets. 
Therefore $S$ is a spectrum involving $\{f\}$ iff it is definable in Presburger arithmetic. 
\end{rem} 

\section{Decidable FO Classes}%Fragments of First-Order Logic}

Church's Theorem states that the set of valid FO sentences (in a vocabulary with at least one binary relation symbol) is undecidable. 
This fact is sometimes expressed by saying that FO logic is undecidable. 
In this section we shall show, thanks to results and tools earlier developed, that certain fragments of FO logic are decidable. 
The strategy is to observe that, sometimes, the problem of determining whether a sentence is satisfiable reduces to the problem of establishing whether it has a finite model. 
When this happens, i.e. when every satisfiable sentence has a finite model, we say that such a class of sentences has the \textbf{finite model property}. 
A set of valid FO sentences $\Phi$ with the finite model property is decidable: by completeness for FO logic $\Phi$ is enumerable; but $$\Phi^c:=\{ \phi : \phi \text{ is not valid}\}=\{\phi: \lnot \phi \text{ has a finite model}\}$$ is enumerable as well, hence $\Phi$ is decidable. 

\subsubsection{L\"owenheim Class}

Let $\sigma=\{ P_1, \ldots, P_r\}$ be a vocabulary with a finite number of unary relations $P_i$. 
As a consequence of Example \ref{unary}, for each $\sigma$-structure $\mathcal{A}$ and $m \ge 1$ there is a $\sigma$-structure $\mathcal{B}$ such that $\mathcal{A} \simeq_m \mathcal{B}$ and $B$ contains at most $m \cdot 2^r$ elements. 
In particular if a $\sigma$-sentence $\phi$ with quantifier rank $m$ is satisfiable, then it is satisfiable already over a domain with at most $m \cdot 2^r$ elements. 
Therefore 

\begin{thm} Let $\sigma=\{ P_1, \ldots, P_r\}$; the set of valid $\sigma$-sentences is decidable.

\end{thm}

\subsubsection{G\"odel-Kalm\'ar-Sch\"utte Class}
Let $\sigma$ be a relational vocabulary. We shall prove that the class of $\forall^2 \exists^{\ast}$-sentences involving $\sigma$ and without the equality symbol has the finite model property. 
We shall express a necessary condition for the satisfiability of $\forall^2 \exists^{\ast}$-sentences and show with a probabilistic argument that it is sufficient for finite satisfiability. 
In fact we shall prove a more general result, requiring additional semantic conditions; these conditions are met in our case.  

Let $\psi$ be $\forall x \forall y \exists z_1 \ldots \exists z_m \phi(x,y,z_1, \ldots, z_m)$ where $\phi$ is a quantifier-free $\sigma$-sentence. 
$\psi$ is equivalent to 
$$\forall x \forall y \exists z_1 \ldots \exists z_m \exists v_1 \ldots \exists v_m(x \neq y \rightarrow \phi(x,x,z_1, \ldots, z_m) \land \phi(x,y,v_1, \ldots, v_m))\text{.}$$ 
Moreover, we can impose inequalities on the variables by means of the equivalence 
$$\exists x \exists y \phi(x,y) \invertmodels \models \exists x \exists y ((\phi(x,y) \lor \phi(x,x)) \land x \neq y)\text{,}$$
so that we can assume that $\psi$ has the form 
\begin{equation} 
\label{godel_sent}
\forall x_1 \forall x_2 \exists x_3 \ldots \exists x_m (x_1 \neq x_2 \rightarrow \phi(x_1, \ldots, x_m) )
\end{equation}
with $\phi(x_1, \ldots, x_m) \models x_i \neq x_j$ for $i \neq j$ and restrict our attention to structures with at least $m$ elements. 
Call such sentences \textbf{G\"odel sentences}. 

\begin{defn} Given a $\sigma$-structure $\mathcal{A}$, $a \in A$ is a \textbf{king} if there is no other element $b$ such that $\phi^0 _a = \phi^0_b$. 
\end{defn}

\begin{lem} Let $\psi$ be a $\sigma$-sentence without equality symbol.
If $\psi$ is satisfiable it has a model with no kings. 
\begin{proof} Suppose $\mathcal{A} \models \psi$. 
Let $\mathcal{A} \times 2$ the structure with domain $A \times \{0,1\}$ and such that for every $k$-ary $R$ in $\sigma$ $R^{A\times 2}:=\{(a_1, i_1), \ldots, (a_k, i_k): R^A a_1 \ldots a_k\}$. 
$\mathcal{A} \times 2 \models \psi$ and $\mathcal{A} \times 2$ has no kings.
\end{proof}
\end{lem}

Let $\mathcal{A}$ be a $\sigma$-structure. 
If $\rho(x_1, \ldots, x_m)$ is a type and $i$, $j$ are distinct numbers not greater than $m$, let $\rho_i(x_1)$ and $\rho_{i,j}(x_1, x_2)$ be the type of elements of $A$ and $A^2$ respectively induced by $\rho$. 
This means that if $\rho=\phi^0 _{\mathbf{a}_m}$, then $\rho_i=\phi^0 _{a_i}$ and $\rho_{i,j}=\phi^0 _{a_i a_j}$. 

\begin{defn} Let $\psi$ and $\phi$ as in \ref{godel_sent}. 
For a $\sigma$-structure $\mathcal{A}$ let $P \subseteq \{\phi^0 _a : a \in A\}$ and $Q\subseteq \{\phi^0 _{ab} : a,b \in A\} $. $P$ and $Q$ satisfy \textbf{G\"odel criterion} for $\phi$ if 
\begin{enumerate} 
\item \label{godel_uno} For all $\gamma$, $\gamma' \in P$ there is $\chi \in Q$ such that $\chi_1 =\gamma$ and $\chi_2 =\gamma'$. 
\item \label{godel_due} Every $\chi \in Q$ can be extended to $\rho(x_1, \ldots, x_m)$ such that 
\begin{enumerate} 
\item $\rho_{1,2}=\chi$;
\item $\rho_{i} \in P$ for all $i=1, \ldots, m$;
\item $\rho_{i,j} \in Q$ for all $i\neq j$ in $\{1, \ldots, m\}$;
\item $\rho \models \phi(x_1, \ldots, x_m)$. 
\end{enumerate}
\end{enumerate} 
\end{defn} 

\begin{lem} Let $\psi$ as in \ref{godel_sent}. 
If $\psi$ has a model $\mathcal{A}$ without kings then there are nonemepty $P$, $Q$ as above satisfying G\"odel criterion for $\phi$. 
\begin{proof} It is enough to efine $P:=\{\phi^0 _a : a \in A\}$ and $Q:=\{\phi^0 _{ab} : a,b \in A \text{ and } a\neq b\} $. 
\end{proof}
\end{lem}

Therefore in this special case G\"odel criterion is automatically fulfilled. 
But it is also sufficient for finite satisfiability, as we now show.  

\begin{thm} Let $\psi$ be a G\"odel sentence and suppose $P$, $Q$ satisfy G\"odel criterion. 
Then $\psi$ admits a finite model.  
\begin{proof} Order the elements of $P$ and suppose $\norm{P}=p$. 
We shall construct probabilistically a labeled structure $\mathcal{A}$ with domain $\{1, \ldots, np\}$, for any $n \ge m$, that satisfies $\psi$. 
\setlist[description]{font=\normalfont}
\begin{description}
\item[\emph{Step 1.}] If $0 \le i <n$ and $1 \le j \le p$, then every number in $\{1, \ldots, np\}$ can be written uniquely in the form $ip+j$. 
Set $\phi^0 _{ip+j}$ the $j$-th member of $P$. 
\item[\emph{Step 2.}] Let $1 \le a < b \le np$. By (\ref{godel_uno}) the set $\{ \chi \in Q : \chi_1=\phi^0 _a , \chi_2 =\phi^0 _b\}$ is nonempty. 
Pick randomly $\chi$ in this set and let $\chi=\phi^0 _{ab}$. 
\item[\emph{Step $3\le j \le m$.}] Decide whether $\mathcal{A} \models R\mathbf{x}_k [\mathbf{a}_k]$ (where $\mathbf{a}_k=(a_1, \ldots, a_k)$ has exactly $j$ distinct components) uniformly at random.  
\item[\emph{Step $m+1, m+2, \ldots$}] Set $\mathcal{A} \models \lnot R \mathbf{x}_k [\mathbf{a}_k] $ (where $\mathbf{a}_k=(a_1, \ldots, a_k)$ has more than $m$ distinct components). 
\end{description}
Now, let $S_n$ be the set of all the outcomes of the above probabilistic experiment. 
Equip $S_n$ with the uniform probability distribution $\pi$. 
Let $s$ be the number of formulae $Rx_{i_1}\ldots x_{i_k}$ with $3 \le k \le m$ and at least three distinct components.
Let $q:=\norm{Q}$, $r:=\binom{m}{2}-1$, $t:=q^{-r} \cdot 2^{-s}$. 

\setcounter{claim}{0}
\begin{claim} Let $\chi \in Q$, and $\rho$ be the extension whose existence is guaranteed by (\ref{godel_due}). 
Let $\mathbf{a}_m=(a_1, \ldots, a_m)$ be distinct elements of $A=\{1, \ldots, np\}$. 
Then $$\pi(\phi^0 _{\mathbf{a}_m}=\rho|\phi^0 _{a_1 a_2}=\chi , \phi^0 _{a_i} =\rho_i ,i=3, \ldots, m) \ge t \text{.}$$ 
\begin{proof} 
There are $s$ formulae $Rx_{i_1} \ldots x_{i_m}$ with at least three distinct variables and the probability that $\mathcal{A}$ satisfies each of these formulae is $2^{-1}$. 
We can choose couples as in \emph{Step 2.} arbitrarily, provided that $\phi^0 _{a_1 a_2}=\chi$, therefore each time we choose $r$ elements from a set of $q$.  
Therefore the probability that we have to compute is at least $t$.
\end{proof}
\end{claim}
Now, let $l:=\left \lfloor{\frac{n-2}{m-2}}\right \rfloor$. 
In particular $n -2\ge l(m-2)$.  
\begin{claim} 
Let $a_1, a_2$ be distinct elements of $A$. Let $H$ be the subclass of $S_n$ of $\sigma$-structures that satisfy $\lnot \exists x_3 \ldots \exists x_m \phi [a_1, a_2] $. 
Then $\pi(H) \le (1-t)^l$.  
\begin{proof} Let $\chi $ be any element of $Q$, and choose $\rho$ in accord with (\ref{godel_due}). 
Let $K$ be the subclass of $S_n$ of $\sigma$-structures that satisfy $\lnot \exists x_3 \ldots \exists x_m \rho [a_1, a_2] $. 
By G\"odel criterion it is enough to prove that $\pi(K|\phi^0 _{a_1 a_2} =\chi)\le (1-t)^l$. $A$ contains at least $n-2 \ge l(m-2)$ distinct elements $a_{i,j} \in \{1, \ldots, np\} \setminus \{a_1, a_2\} $ such that $\phi^0 _{a_{i,j}} =\rho_j$ for $i=1, \ldots, l$ and $j=3, \ldots, m$. 
The events `$\mathcal{A} \models \rho[a_1, a_2, a_{i,3}, \ldots, a_{i,m}]$' are $l$ (because $i=1, \ldots, l$) and are independent. 
By Claim 1 they have probability at least $t$. 
Therefore $\pi(K|\phi^0 _{a_1 a_2} =\chi)\le (1-t)^l$. 
\end{proof} 
\end{claim}
Let $M$ be the the subclass of $S_n$ of $\sigma$-structures that satisfy $\lnot \phi$. 
$\pi(M)\le \sum_{a_1 \neq a_2} \pi(H) \le np (np-1)(1-t)^{l}  \to 0$ when $n \to \infty$. 
Therefore the probability that $\mathcal{A} \in S_n$ satisfies $\psi$ is positive for $n$ sufficiently large. 
\end{proof}
\end{thm} 

\section{Word Models}
In this final section we turn our attention to regular languages and FO logic; regular languages are those strings that a finite state automaton accepts. 
Is there a way to characterize this class of strings in purely logical terms? 
The answer is no, if with logic we mean FO logic, as we did.  
There are special languages, however, that are exactly those definable (in a sense made clear later) in FO logic. 
This is going to be established using some of the finite model theoretic apparatus developed so far. 

An \textbf{alphabet} is just a finite a set $S$. 
A finite-length sequence of elements of $S$ is called \textbf{string} or \textbf{word}. 
If $x$ is a string we write $\len{x}$ for its length. 
$\epsilon$ is the unique string over $S$ such that $\len{\epsilon}=0$ and is called the \textbf{empty string}. 
We denote by $S^{\ast}$ the set of strings over $S$. 
A \textbf{language} is a subset of $S^{\ast}$. 
We define the following operations on languages: 
\begin{align*} 
A \cup B &:=\{x \in S^{\ast} : x \in A \text{ or } x \in B \} \\
AB &:=\{ xy \in S^{\ast} : x \in A, y \in B\} \\ 
\compl A&:=\{x \in S^{\ast} : x \notin A\} \\
A^{\ast} &:= \bigcup_{n \in \omega} A^n \text{.}
\end{align*}

\begin{defn} A \textbf{regular} language over $S$ is defined inductively as follows:
\begin{itemize}
\item the empty set $\emptyset$ is regular;
\item for every $a \in S$ the language $\{a\}$ is regular;
\item if $A$ and $B$ are regular, then $\compl A$\footnote{One can actually show that $\sim$ is redundant, but we include it here for the sake of convenience.}, $A \cup B$ and $AB$ are regular;
\item if $A$ is regular, then $A^{\ast}$ is regular.
%\item no other language over $S$ is regular.
\end{itemize} 
\end{defn}
By convention, $\emptyset^{\ast}=\{\epsilon\}$, which is therefore a regular language. 

\subsubsection{Word Models}
Let $S$ be an alphabet and let $\tau_S$ be the vocabulary $\{<\} \cup \{P_s : s \in S\}$, where $<$ is a binary and each $P_i$ is a unary relation symbol. 
Given a string $v=s_{1} \cdots s_{n} \in S^{\ast}$ we define a labeled $\{<\}$-structure $\mathcal{A}$ such that $\norm{A}=\len{v}$ with in addition $P^A _s:=\{i \in A : s_i=s\}$. 
$P^A _s$ collect for each letter $s$ the positions of $v$ which carry $s$. 
$\mathcal{A}_v=(A_v, <, (P_s)_{s\in S})$ is said to be the \textbf{word model} for $v$. 
A fundamental remark is that if $v, w \in S^{\ast}$, a word model for $vw$ is given by $\mathcal{A}_v \triangleleft \mathcal{A}_w$.  

Hopefully, an example will clarify the situation. 
Let $S=\{a,b\}$ and $v=abbab$. Then the word model is $A=\{1,2,3,4,5\}$ equipped with the natural ordering $<$, and $P^A _a=\{1,4\}$, $P^A _b=\{2,3,5\}$.

\begin{defn} Given an alphabet $S$, a language $L \subseteq S^{\ast}$ is \textbf{definable} in FO logic, if there exists a $\tau_S$-sentence $\phi$ such that $\Mod(\phi)=\{\mathcal{A}_v : v \in L\}$.  
\end{defn}

By convention, the language $\{\epsilon\}$ is defined by 
$\forall x (x=x)$ and the language $\emptyset$ by $\exists x (x \neq x)$. $S^{\ast} \setminus \{\epsilon\}$ is defined, modulo the axioms of strict orderings, by the sentence $\phi_{\ast}$:
$$\forall x \bigvee_{a \in S} P_a x \land \bigwedge_{a \neq b} \forall x \lnot(P_a x \land P_b x)\text{.}$$ 

Not every regular language is definable in FO logic. 
Let $S=\{a\}$. 
$L=\{a^n : n \text{ is even}\}$ is clearly regular, but as a consequence of Example \ref{even_order} the strings $a^{2^m}$ and $a^{2^m+1}$ cannot be told apart by a sentence of quantifier rank $m$, for any $m$. 

We turn to the problem of characterizing the languages definable in FO logic.

\subsubsection{Star-Free Languges} 
We saw that the class of regular languages does not overlap with the class of FO-definable languages. 

\begin{rem}
Let $\psi_{\bot}(x) $ be the sentence $\forall y (x < y \lor x=y)$ and $\psi_{\top}(x)$ be $\forall y(y <x \lor x=y)$. 
Then 
\begin{itemize}
\item the empty set $\emptyset$ is definable (by definition);
\item for every $a \in S$ the language $\{a\}$ is definable by $$\phi_{\ast} \land \exists x \forall y(y=x \land P_a x)\text{;}$$
\item if $A$ and $B$ are definable by $\phi_A$ and $\phi_B$, then $\compl A$, $A \cup B$ and $AB$ are definable by $\phi_{\ast} \land \lnot \phi_{A}$, $\phi_{\ast} \land (\phi_{A} \lor \phi_B)$, and $$\phi_{\ast} \land \exists x \exists y \exists z (\psi_{\bot}(x) \land \psi_{\top}(z) \land \forall u (x \le u \le y \rightarrow \phi_{A}) \land \forall u (y < u \le z \rightarrow \phi_{B}))$$
respectively. 
\end{itemize}
However, the difficulties that arise when trying to define $A^{\ast}$, cannot be overcome, as we shall now prove.  
\end{rem}

\begin{defn} A \textbf{star-free} regular language over $S$ is defined inductively as follows:
\begin{itemize}
\item the empty set $\emptyset$ is star-free regular;
\item for every $a \in S$ the language $\{a\}$ is star-free regular;
\item if $A$ and $B$ are regular, then $\compl A$, $A \cup B$ and $AB$ are star-free regular.
%\item no other language over $S$ is star-free regular.
\end{itemize} 
\end{defn}

\begin{thm} A language is star-free regular iff it is definable in FO logic.
\begin{proof} One direction has been established in the previous remark. 

To prove the other direction, we expand the vocabulary $\tau_S$ by adding the constant symbol $\{\bot\}$. We are interested only in models of $\phi_{\ast} \land \psi_{\bot}(\bot)$. 
We prove by induction on $\phi$ that if $\{(A_v, \bot^{A_v}) : v \in L\}=\Mod(\phi_{\ast} \land \psi_{\bot}(\bot) \land \phi)$, then $L$ is star-free regular. 

If $\phi$ is atomic then either $\phi$ is $\bot=\bot$ or $P_a\bot$ for some $a \in S$. 
In the first case let $L=S^{\ast}$, in the second $L=\{a\}S^{\ast}$. 
The argument for negation or disjunction is straightforward.
Assume $\phi$ is $\exists x \psi (x)$. Clearly 
\begin{multline*}
\Mod(\phi_{\ast} \land \psi_{\bot}(\bot) \land \exists x \psi (x))=\\ \Mod(\phi_{\ast} \land \psi_{\bot}(\bot) \land \psi (\bot)) \cup \Mod(\phi_{\ast} \land \psi_{\bot}(\bot) \land \exists x (x\neq \bot \land \psi (x)))\text{.}
\end{multline*}
The first member of the union, by inductive hypothesis, describes a star-free language. 
We can think of the second member as the class of finite $\tau_S \cup \{\bot, c\}$-structures $\mathcal{A}$ such that 
$(A, \bot^A, c^A) \models \phi_{\ast} \land \psi_{\bot}(\bot) \land c\neq \bot \land \psi (c)$. 
Such structures can be written as $\mathcal{A} = \mathcal{A}_1 \triangleleft \mathcal{A}_2$ where $(A_1, \bot^{A_1}) \models \phi_{\ast} \land \psi_{\bot}(\bot)$ and $(A_2, c^{A_2}) \models \phi_{\ast} \land \psi_{\bot}(c)$. 
Now, let $\phi^m _{\mathcal{A}_1}$ and $\phi^m _{\mathcal{A}_2}$  be the suitable $m$-types. 
By Remark \ref{sums} the fact that $$\mathcal{A}_1 \models \phi_{\ast} \land \psi_{\bot}(\bot) \land \phi^m _{\mathcal{A}_1} \quad \text{ and } \quad \mathcal{A}_2 \models \phi_{\ast} \land \psi_{\bot}(c) \land \phi^m _{\mathcal{A}_2}$$ implies that $\mathcal{A}_1 \triangleleft \mathcal{A}_2 \models \psi(c)$. 
But by inductive hypothesis 
$$\phi_{\ast} \land \psi_{\bot}(\bot) \land \phi^m _{\mathcal{A}_1} \quad \text{ and } \quad  \phi_{\ast} \land \psi_{\bot}(c) \land \phi^m _{\mathcal{A}_2}$$ define star-free languages $L_1$, $L_2$. 
Take $L:=L_1L_2 $ to complete the proof. 
\end{proof}
\end{thm}
